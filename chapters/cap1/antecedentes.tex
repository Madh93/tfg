%%%%%%%%%%%%%%%%%%%%%%%%%%%%%%%%%%%%%%%%%%%%%%%%%%%%%%%%%%%%%%%%%%%%%%%%%%%%%%%%
% Chapter 1: Introducción
%%%%%%%%%%%%%%%%%%%%%%%%%%%%%%%%%%%%%%%%%%%%%%%%%%%%%%%%%%%%%%%%%%%%%%%%%%%%%%%%

%+++++++++++++++++++++++++++++++++++++++++++++++++++++++++++++++++++++++++++++++
% \section{Antecedentes}
% \label{1:sec:1}

Vivimos en una época emocionante para la ciencia y la ingeniería. Ya desde hace
200 años con el comienzo de la revolución industrial, se empezaron a cimentar
las primeras bases de la automatización de todas aquellas pequeñas tareas
repetitivas que eran un lastre para los tiempos de producción.

En el siglo XX la auténtica revolución ha llegado a través de la computación e
internet. La automatización en la computación ha sido, y es crucial, para el
desarrollo de nuevas líneas de trabajo como la inteligencia artificial.
Precisamente la inteligencia artificial y su implementación en la robótica ha
sido el último gran paso en los avances de la humanidad para optimizar,
sustituir o eliminar todo aquel trabajo tedioso, repetitivo o simplemente,
peligroso. Pero para ello se requiere del uso de todo tipo de sensores que
permitan simular el comportamiento de un ser humano.

La visión artificial es uno de los campos de investigación que más interés han
causado en las últimas décadas. Sin embargo, no ha sido hasta hace unos pocos
años cuando se ha empezado a conseguir los resultados esperados durante todo
este tiempo. El uso de cámaras permite obtener mucha y variada información
acerca del entorno de un robot: objetivos, obstáculos e incluso muchos datos que
aportan información directa de la situación que se visualiza.

%+++++++++++++++++++++++++++++++++++++++++++++++++++++++++++++++++++++++++++++++
