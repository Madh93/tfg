%%%%%%%%%%%%%%%%%%%%%%%%%%%%%%%%%%%%%%%%%%%%%%%%%%%%%%%%%%%%%%%%%%%%%%%%%%%%%%%%
% Chapter 1: Introducción
%%%%%%%%%%%%%%%%%%%%%%%%%%%%%%%%%%%%%%%%%%%%%%%%%%%%%%%%%%%%%%%%%%%%%%%%%%%%%%%%

%+++++++++++++++++++++++++++++++++++++++++++++++++++++++++++++++++++++++++++++++
% \section{Objetivo}
% \label{1:sec:3}

El tema central de este proyecto es muy ambicioso, de este modo, es necesario
poder realizar una distinción entre el objetivo principal y los objetivos
específicos que lo rodean.

%--------------------------------------
\subsection{Objetivo general}

\textcolor{red}{¿Detección y esquiva o reconstrucción del mapa?}

El objetivo principal de este Trabajo de Fin de Grado es poder integrar el uso
de un sistema estereoscópico de dos cámaras en un robot, para la detección y
posteriormente esquiva de los obstáculos que se encuentren mediante la
construcción de un mapa tridimensional como referencia.

%--------------------------------------
\subsection{Objetivos específicos}

Los objetivos específicos que componen el proyecto son:
% \newline
\begin{itemize}
  \item Uso de una cámara comercial de entretenimiento en un proyecto de
  investigación.
  \item Reconstrucción de un mapa tridimensional a partir de las imágenes
  recogidas por las cámaras.
  \item Integración de cámaras estereoscópicas junto a otros sistemas de
  detección de obstáculos.
  \item Combinación de odometría mecánica y odometría láser.
\end{itemize}

%+++++++++++++++++++++++++++++++++++++++++++++++++++++++++++++++++++++++++++++++
