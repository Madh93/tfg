%%%%%%%%%%%%%%%%%%%%%%%%%%%%%%%%%%%%%%%%%%%%%%%%%%%%%%%%%%%%%%%%%%%%%%%%%%%%%%%%
% Chapter 1: Introducción
%%%%%%%%%%%%%%%%%%%%%%%%%%%%%%%%%%%%%%%%%%%%%%%%%%%%%%%%%%%%%%%%%%%%%%%%%%%%%%%%

%+++++++++++++++++++++++++++++++++++++++++++++++++++++++++++++++++++++++++++++++
% \section{Objetivo}
% \label{1:sec:3}

El tema central de este proyecto es muy ambicioso, de este modo, es necesario
poder realizar una distinción entre el objetivo principal y los objetivos
específicos que lo rodean.

%+++++++++++++++++++++++++++++++++++++++++++++++++++++++++++++++++++++++++++++++
\subsection{Objetivo general}
El objetivo principal de este Trabajo de Fin de Grado es poder integrar el uso
de un sistema estereoscópico de dos cámaras en un robot, para la construcción de
un mapa en 3D y su posterior localización en el mismo.

%+++++++++++++++++++++++++++++++++++++++++++++++++++++++++++++++++++++++++++++++
\subsection{Objetivos específicos}
Los objetivos específicos que componen el proyecto son:
\begin{itemize}
  \item Uso de una cámara comercial de entretenimiento en un proyecto de
  investigación.
  \item Integración de cámaras estereoscópicas junto a otros sistemas de
  detección de obstáculos.
  \item Combinación de odometría mecánica y odometría láser.
  \item Reconstrucción de un mapa tridimensional a partir de las imágenes
  recogidas por las cámaras.
  \item Localización en un mapa tridimensional construido previamente.
\end{itemize}

%+++++++++++++++++++++++++++++++++++++++++++++++++++++++++++++++++++++++++++++++
