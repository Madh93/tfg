%%%%%%%%%%%%%%%%%%%%%%%%%%%%%%%%%%%%%%%%%%%%%%%%%%%%%%%%%%%%%%%%%%%%%%%%%%%%%
% Chapter 6: Conclusions and future work lines
%%%%%%%%%%%%%%%%%%%%%%%%%%%%%%%%%%%%%%%%%%%%%%%%%%%%%%%%%%%%%%%%%%%%%%%%%%%%%%%

%+++++++++++++++++++++++++++++++++++++++++++++++++++++++++++++++++++++++++++++++
% \section{Conclusions}
% \label{6:sec:1}

In this project we have seen that stereoscopic cameras allow to obtain images of
the world very close to reality. Use only visual odometry of these cameras works
fine in most situations both indoors and in more open places, being a clear
competitor to the current system of the Perenquén project, using a camera RGB-D.
Furthermore, it is important to mention that the integration of mechanical
sensors and laser sensors allow to correct some of the problems that are likely
to happen when the lighting conditions are not ideal.

It is need to remember, it is very difficult to collect accurate information
from nature, the environment around us is alive, however, stereo vision is the
key, along with other sensors to obtain the best possible approach of the world.

%+++++++++++++++++++++++++++++++++++++++++++++++++++++++++++++++++++++++++++++++
