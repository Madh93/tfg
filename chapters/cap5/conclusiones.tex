%%%%%%%%%%%%%%%%%%%%%%%%%%%%%%%%%%%%%%%%%%%%%%%%%%%%%%%%%%%%%%%%%%%%%%%%%%%%%
% Chapter 5: Conclusiones y Líneas Futuras 
%%%%%%%%%%%%%%%%%%%%%%%%%%%%%%%%%%%%%%%%%%%%%%%%%%%%%%%%%%%%%%%%%%%%%%%%%%%%%%%

%+++++++++++++++++++++++++++++++++++++++++++++++++++++++++++++++++++++++++++++++
% \section{Conclusiones}
% \label{5:sec:1}

En este trabajo se ha visto que las cámaras estereoscópicas permiten obtener
unas imágenes del mundo que le rodea muy próximas a la realidad. Por sí sola, la
odometría visual de este tipo de cámaras funciona muy bien en la mayoría de
situaciones, tanto en espacios cerrados como en lugares más abiertos, siendo un
claro competidor del sistema actual de la silla Perenquén, el uso de una cámara
RGB-D. Por otro lado, es importante mencionar que la integración de sensores
mecánicos y sensores láser permite corregir algunos de los problemas que son
propensos a ocurrir cuando las ocasiones lumínicas no son las idóneas.

Es necesario recordar, que es muy difícil recoger con exactitud la información
de la naturaleza, el entorno que nos rodea está vivo, sin embargo, la visión
estereoscópica sirve de pilar, junto con otros sensores para obtener la mayor
aproximación posible del mundo.

%+++++++++++++++++++++++++++++++++++++++++++++++++++++++++++++++++++++++++++++++
