%%%%%%%%%%%%%%%%%%%%%%%%%%%%%%%%%%%%%%%%%%%%%%%%%%%%%%%%%%%%%%%%%%%%%%%%%%%%%%%%
% Chapter 4: Desarrollo
%%%%%%%%%%%%%%%%%%%%%%%%%%%%%%%%%%%%%%%%%%%%%%%%%%%%%%%%%%%%%%%%%%%%%%%%%%%%%%%%

%+++++++++++++++++++++++++++++++++++++++++++++++++++++++++++++++++++++++++++++++
% \section{Preparación del entorno}
% \label{4:sec:1}

Como se pudo observar en el capítulo ~\ref{chapter:recursos}, el desarrollo de
este trabajo se realiza en base a ROS, concretamente la versión 'Indigo'. Es
necesario instalar y configurar un directorio de trabajo para ROS donde estará
alojado el paquete con el código fuente del proyecto.

El sistema operativo que se ha utilizado ha sido Ubuntu 14.04 LTS (64-bit), una
distribución GNU/Linux muy sencilla de utilizar para la que se distribuyen los
paquetes que conforman ROS. Cabe mencionar, que todos los pasos que se
describen a continuación son perfectamente válidos para versiones más recientes
de Ubuntu o distribuciones derivadas, solo es necesario modificar algunos
detalles como la versión o el nombre de la distribución. 

%+++++++++++++++++++++++++++++++++++++++++++++++++++++++++++++++++++++++++++++++
\subsection{Instalación y configuración de ROS}
% http://wiki.ros.org/es/hydro/Installation/Ubuntu
% http://wiki.ros.org/ROS/Tutorials

%--------------------------------------
\paragraph{Instalar ROS} \hspace{0pt}

ROS no se encuentra en los paquetes oficiales de Ubuntu, es necesario añadir un
repositorio externo, el oficial de ROS. Para ello se introduce lo siguiente en
una terminal:
\\
\begin{lstlisting}
  $ sudo sh -c 'echo "deb http://packages.ros.org/ros/ubuntu $(lsb_release -sc) main" > /etc/apt/sources.list.d/ros-latest.list'
\end{lstlisting}

Se añaden las claves del repositorio externo:
\\
\begin{lstlisting}
  $ sudo apt-key adv --keyserver hkp://ha.pool.sks-keyservers.net --recv-key 0xB01FA116
\end{lstlisting}

Con esto solamente es necesario actualizar la base de datos de los paquetes
disponibles en los diferentes repositorios:
\\
\begin{lstlisting}
  $ sudo apt-get update
\end{lstlisting}

El gestor de paquetes ya puede encontrar los paquetes de ROS entre sus
repositorios. Los paquetes se pueden instalar de forma individual, pero es
recomendable realizar una instalación completa para no echar en falta ningún
paquete en el futuro. Para ello en la terminal instalamos el siguiente paquete:
\\
\begin{lstlisting}
  $ sudo apt-get install ros-indigo-desktop-full
\end{lstlisting}

El paquete 'ros-indigo-desktop-full' viene con muchas de las herramientas que se
han usado en el proyecto: Rviz, Rtabmap, Rqt, etc. Aunque no incluye la
herramienta 'rosinstall' la cual permite instalar paquetes de ROS
independientemente del sistema:
\\
\begin{lstlisting}
  $ sudo apt-get install python-rosinstall
\end{lstlisting}

%--------------------------------------
\paragraph{Configuración de ROS} \hspace{0pt}

Antes de poder utilizar ROS es necesario configurar 'rosdep'. Esta herramienta
permite instalar fácilmente dependencias del sistema que surgen cuando se
requiere compilar el código fuente de un paquete ROS. Es necesario introducir en
una terminal lo siguiente:
\\
\begin{lstlisting}
  $ sudo rosdep init
  $ rosdep update
\end{lstlisting}

Por último, también es conveniente cargar las variables del entorno de ROS. Si
se utiliza Bash como shell basta con introducir lo siguiente:
\\
\begin{lstlisting}
  $ echo "source /opt/ros/indigo/setup.bash" >> ~/.bashrc
  $ source ~/.bashrc
\end{lstlisting}

En caso de utilizar otra shell por defecto (por ejemplo Zsh), en 
'/opt/ros/indigo' también se cuenta con el script equivalente, cuyo contenido es
necesario volcar en el archivo de configuración de la shell. Con esto último,
todas las variables de entorno se cargan automáticamente al abrir una nueva
shell.

%--------------------------------------
\paragraph{Crear un workspace} \hspace{0pt}

Un workspace es un directorio donde se trabaja con paquetes catkin (los paquetes
oficiales utilizados por ROS), pudiendo modificar, compilar o añadir nuevos
paquetes.

Al igual que en el entorno real, se puede crear un directorio en el directorio
personal del usuario que va a contener el workspace:
\\
\begin{lstlisting}
  $ mkdir -p ~/ROS/ws/src
  $ cd ~/ROS/ws/src
  $ catkin_init_workspace
\end{lstlisting}

Con el último comando, se inicia el workspce, creando un archivo de
instrucciones de Cmake. Por último, es necesario compilar estas instrucciones:
\\
\begin{lstlisting}
  $ cd ~/ROS/ws
  $ catkin_make
\end{lstlisting}

Tras finalizar la compilación, el directorio del workspace estará estructurado
como un workspace válido. Al igual que es necesario añadir a los archivos de
configuración de la shell las variables de ROS, también es recomendable añadir
los del workspace:
\\
\begin{lstlisting}
  $ echo "source ~/ROS/ws/devel/setup.bash" >> ~/.bashrc
  $ source ~/.bashrc
\end{lstlisting}

%--------------------------------------
\paragraph{Crear un paquete} \hspace{0pt}

Es posible crear un nuevo paquete catkin en un workspace existente, para ello es
necesaria escribir en una terminal lo siguiente:
\\
\begin{lstlisting}
  $ cd ~/ROS/ws/src
  $ catkin_create_pkg myps4eye
\end{lstlisting}

En este caso se ha creado un paquete llamado 'myps4eye'. Con este último comando
se creará una carpeta nueva con los archivos de configuración necesarios para
compilar el paquete. Si se modifica el código fuente, en un paquete del
workspace, es necesario compilarlo de la siguiente forma:
\\
\begin{lstlisting}
  $ cd ~/ROS/ws
  $ catkin_make
\end{lstlisting}

%--------------------------------------
\paragraph{Sistema de archivos} \hspace{0pt}

Una de las mayores ventajas que tiene ROS está en el sistema de archivos. Aunque
los proyectos y los paquetes están almacenados en directorios, existen una serie
de comandos muy útiles para acceder, o modificar rápidamente paquetes.

Mostrar archivos:
\\
\begin{lstlisting}
  $ rosls [ruta[/subdirectorio]]
\end{lstlisting}

Cambiar de directorio:
\\
\begin{lstlisting}
  $ roscd [ruta[/subdirectorio]]
\end{lstlisting}

Editar archivo de un paquete:
\\
\begin{lstlisting}
  $ rosed [paquete] [archivo]
\end{lstlisting}

Buscar paquetes:
\\
\begin{lstlisting}
  $ rospack find [paquete]
\end{lstlisting}


%+++++++++++++++++++++++++++++++++++++++++++++++++++++++++++++++++++++++++++++++
\subsection{Instalación y configuración de PlayStation Camera}
% https://github.com/longjie/ps4eye

La PlayStation Camera solamente tiene soporte oficial para la consola
PlayStation 4 de Sony. Sin embargo varios desarrolladores de forma independiente
han trabajado en hacer funcionar la cámara en PC. Por otra parte también se ha
desarrollado un paquete no oficial para ROS que permite la utilización de la
cámara como si de una cámara estéreo más se tratase.

%--------------------------------------
\paragraph{Requisitos} \hspace{0pt}

% http://ps4eye.tumblr.com/
En primer lugar se necesita el modelo CUH-ZEY1J de Playstation Camera (hasta el
momento el único modelo disponible en el mundo). La principal dificultad con la
que cuenta es que es necesario alterar el cable de la cámara para modificar el
puerto propietario por un puerto USB 3.0 estándar.

En segundo lugar es necesario utilizar una versión de kernel superior a la
3.17.3. Para descargar esta versión concreta podemos realizar las siguientes
acciones en una terminal:
\\
\begin{lstlisting}
  $ wget http://kernel.ubuntu.com/~kernel-ppa/mainline/v3.17.3-vivid/linux-headers-3.17.3-031703_3.17.3-031703.201411141335_all.deb
  $ wget http://kernel.ubuntu.com/~kernel-ppa/mainline/v3.17.3-vivid/linux-headers-3.17.3-031703-generic_3.17.3-031703.201411141335_amd64.deb
  $ wget http://kernel.ubuntu.com/~kernel-ppa/mainline/v3.17.3-vivid/linux-image-3.17.3-031703-generic_3.17.3-031703.201411141335_amd64.deb
\end{lstlisting}

Y para instalar:
\\
\begin{lstlisting}
  $ sudo dpkg -i linux-headers-3.17.3*.deb linux-image-3.17.3*.deb
\end{lstlisting}

También es necesario instalar el paquete Pyusb para que funcione correctamente:
\\
\begin{lstlisting}
  $ which pip > /dev/null 2>&1 || sudo apt-get install python-pip
  $ sudo pip install --pre pyusb
\end{lstlisting}

%--------------------------------------
\paragraph{Instalación} \hspace{0pt}

Para instalar el paquete de ROS necesitamos clonar el repositorio dle proyecto
que está alojado en Github:
\\
\begin{lstlisting}
  $ cd ~/ROS/ws
  $ git clone https://github.com/longjie/ps4eye.git
\end{lstlisting}

Se necesitan añadir las reglas correspondientes de la cámara en el control de
dispositivos de Linux (udev). Para ello ya podemos trabajar con ROS para
interactuar con el paquete:
\\
\begin{lstlisting}
  $ rosrun ps4eye create_udev_rules
\end{lstlisting}

Una vez hecho, si se conecta la cámara al ordenador mediante un puerto USB 3.0
se debería poder visualizar la cámara con cualquier aplicación de captura de
cámara web. Para comprobar el funcionamiento ejecutamos el siguiente launch del
paquete:
\\
\begin{lstlisting}
  $ roslaunch ps4eye stereo.launch DEVICE:=/dev/video0 viewer:=true
\end{lstlisting}

En este caso se supone que la cámara está en '/dev/video0', aunque si se utiliz
un portátil u otra cámara es necesario que corresponda con otro dispositivo. Se
debería visualizar la cámara izquierda y la cámara derecha junto con una imagen
de disparidad generada a partir de ambas imágenes.

%+++++++++++++++++++++++++++++++++++++++++++++++++++++++++++++++++++++++++++++++
