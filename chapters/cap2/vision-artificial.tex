%%%%%%%%%%%%%%%%%%%%%%%%%%%%%%%%%%%%%%%%%%%%%%%%%%%%%%%%%%%%%%%%%%%%%%%%%%%%%%%%
% Chapter 2: Conceptos
%%%%%%%%%%%%%%%%%%%%%%%%%%%%%%%%%%%%%%%%%%%%%%%%%%%%%%%%%%%%%%%%%%%%%%%%%%%%%%%%

%++++++++++++++++++++++++++++++++++++++++++++++++++++++++++++++++++++++++++++++
% \section{Visión artificial}
% \label{2:sec:1}
% https://campusvirtual.ull.es/1516/course/view.php?id=202
% https://es.wikipedia.org/wiki/Visi%C3%B3n_artificial
% https://en.wikipedia.org/wiki/Computer_vision
% J. GONZÁLEZ JIMÉNEZ, "Visión Por Computador", Editorial Paraninfo. 2000

La visión artificial por computador, es la disciplina científica que se basa en
la adquisición, procesamiento y análisis de las imágenes que se toman del mundo
real, con el objetivo de obtener información relevante acerca de ellas:
detección de objetos, seguimiento del movimiento, reconocimiento de eventos,
etc. Un ejemplo que podemos ver en nuestro día a día, es la detección de caras
en una escena capturada por una cámara digital o smartphone, mediante el uso de
téncicas de reconocimiento de patrones.

Al igual que sucede en otras áreas de la inteligencia artificial, la visión
artificial tiene como objetivo principal obtener la información explícita y el
significado de la realidad de la misma manera que lo haría un ser biológico.

El avance progresivo del hardware con nuevos procesadores digitales de señales
(DSP) y unidades de procesamiento gráfico (GPU), junto con nuevas tecnologías y
planteamientos de cómputo como la computación paralela, ha permitido que en los
últimos años se haya podido implementar nuevos algortimos más rápidos y
eficientes, necesarios para ser utilizados en ámbitos críticos, como sistemas
en tiempo real.

%+++++++++++++++++++++++++++++++++++++++++++++++++++++++++++++++++++++++++++++++
\subsection{Objetivos}

\textcolor{red}{Lorem ipsum dolor sit amet, consectetur adipisicing elit, sed do
 eiusmod tempor incididunt ut labore et dolore magna aliqua. Ut enim ad minim 
 veniam, quis nostrud exercitation ullamco laboris nisi ut aliquip ex ea commodo 
 consequat. Duis aute irure dolor in reprehenderit in voluptate velit esse cillum
 dolore eu fugiat nulla pariatur. Excepteur sint occaecat cupidatat non proident,
 sunt in culpa qui officia deserunt mollit anim id est laborum.}

\textcolor{red}{Lorem ipsum dolor sit amet, consectetur adipisicing elit, sed do
 eiusmod tempor incididunt ut labore et dolore magna aliqua. Ut enim ad minim 
 veniam, quis nostrud exercitation ullamco laboris nisi ut aliquip ex ea commodo 
 consequat. Duis aute irure dolor in reprehenderit in voluptate velit esse cillum
 dolore eu fugiat nulla pariatur. Excepteur sint occaecat cupidatat non proident,
 sunt in culpa qui officia deserunt mollit anim id est laborum.}

%+++++++++++++++++++++++++++++++++++++++++++++++++++++++++++++++++++++++++++++++
\subsection{Dificultades}
La capacidad visual es uno pilares de la inteligencia humana. Su implementación
en la rotótica supone también un importante avance en la inteligencia
artificial. Sin embargo, mientras que la percepción visual es algo innato y
cotidiano para nosotros, la visión artificial es muy compleja y conlleva muchas
dificultades. Entre las principales dificultades, destacan:

\begin{itemize}
  \item \textbf{Mundo tridimensional:} mientras que las imágenes que se
  obtienen con una cámara son bidimensionales, el mundo que nos rodea no. Es
  necesario realizar las transformaciones correspondientes para obtener valores
  correctos.
  \item \textbf{Zonas de interés:} se necesita extraer elementos de información 
  sutiles en imágenes complejas, por lo que entre tanta información es
  necesario reconocer formas, colores, etc.
  \item \textbf{Carácter dinámico de las escenas:} el mundo está vivo, por lo
  que en las imágenes que se toman muchos elementos están en movimiento. Por 
  otro lado, otros factores como luminosidad, contraste, foco... pueden marcar
  una importante diferencia, y por desgracia, estos factores son variables, no
  se pueden controlar.
\end{itemize}

%+++++++++++++++++++++++++++++++++++++++++++++++++++++++++++++++++++++++++++++++
% \subsection{Reconocimiento}

% Detección de objetos

%+++++++++++++++++++++++++++++++++++++++++++++++++++++++++++++++++++++++++++++++
% \subsection{Motion Analysis}

% Video grabación de imágenes de seguridad

%+++++++++++++++++++++++++++++++++++++++++++++++++++++++++++++++++++++++++++++++
% \subsection{Reconstrucción}

%++++++++++++++++++++++++++++++++++++++++++++++++++++++++++++++++++++++++++++++
