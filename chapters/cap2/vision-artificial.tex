%%%%%%%%%%%%%%%%%%%%%%%%%%%%%%%%%%%%%%%%%%%%%%%%%%%%%%%%%%%%%%%%%%%%%%%%%%%%%%%%
% Chapter 2: Conceptos
%%%%%%%%%%%%%%%%%%%%%%%%%%%%%%%%%%%%%%%%%%%%%%%%%%%%%%%%%%%%%%%%%%%%%%%%%%%%%%%%

%++++++++++++++++++++++++++++++++++++++++++++++++++++++++++++++++++++++++++++++
% \section{Visión artificial}
% \label{2:sec:1}
% https://campusvirtual.ull.es/1516/course/view.php?id=202
% https://es.wikipedia.org/wiki/Visi%C3%B3n_artificial
% https://en.wikipedia.org/wiki/Computer_vision
% J. GONZÁLEZ JIMÉNEZ, "Visión Por Computador", Editorial Paraninfo. 2000

La visión artificial por computador, es la disciplina científica que se basa en
la adquisición, procesamiento y análisis de las imágenes que se toman del mundo
real, con el objetivo de obtener información relevante acerca de ellas:
detección de objetos, seguimiento del movimiento, reconocimiento de eventos, etc
\cite{VisionArtificial}. Un ejemplo que podemos ver en nuestro día a día, es la
detección de caras en una escena capturada por una cámara digital o smartphone,
mediante el uso de técnicas de reconocimiento de patrones.

Al igual que sucede en otras áreas de la inteligencia artificial, la visión
artificial tiene como objetivo principal obtener la información explícita y el
significado de la realidad de la misma manera que lo haría un ser biológico.

El avance progresivo del hardware con nuevos procesadores digitales de señales
(DSP) y unidades de procesamiento gráfico (GPU) \cite{VisionArtificialGPU},
junto con nuevas tecnologías y planteamientos de cómputo como la computación
paralela, ha permitido que en los últimos años se haya podido implementar nuevos
algoritmos más rápidos y eficientes, necesarios para ser utilizados en ámbitos
críticos, como sistemas en tiempo real.

%+++++++++++++++++++++++++++++++++++++++++++++++++++++++++++++++++++++++++++++++
\subsection{Dificultades} La capacidad visual es uno pilares de la inteligencia
humana. Su implementación en la robótica supone también un importante avance en
la inteligencia artificial. Sin embargo, mientras que la percepción visual es
algo innato y cotidiano para nosotros, la visión artificial es muy compleja y
conlleva muchas dificultades \cite{VisionPorComputador}. Entre las principales
dificultades, destacan:

\begin{itemize}
  \item \textbf{Mundo tridimensional:} mientras que las imágenes que se
  obtienen con una cámara son bidimensionales, el mundo que nos rodea no. Es
  necesario realizar las transformaciones correspondientes para obtener valores
  correctos.
  \item \textbf{Zonas de interés:} se necesita extraer elementos de información 
  sutiles en imágenes complejas, por lo que entre tanta información es
  necesario reconocer formas, colores, etc.
  \item \textbf{Carácter dinámico de las escenas:} el mundo está vivo, por lo
  que en las imágenes que se toman muchos elementos están en movimiento. Por 
  otro lado, otros factores como luminosidad, contraste, foco... pueden marcar
  una importante diferencia, y por desgracia, estos factores son variables, no
  se pueden controlar.
\end{itemize}

%+++++++++++++++++++++++++++++++++++++++++++++++++++++++++++++++++++++++++++++++
\subsection{Aplicaciones}
La visión artificial resulta de gran utilidad en diferentes áreas de
aplicación, tanto en acciones repetitivas como peligrosas:

\begin{itemize}
  \item \textbf{Inspección y ensamblaje industrial:} el proyecto "Randon Bin
  Picking" (RBP) \cite{AplicacionesRBP} hace uso de visión estéreo para la
  búsqueda de piezas entre objetos de todo tipo para su rápida recuperación.
  \item \textbf{Apoyo en el diagnóstico médico:} en las últimas décadas la
  visión artificial se ha hecho un importante en la medicina para detectar,
  analizar y reconstruir la información obtenida \cite{AplicacionesDiagnostico}.
  \item \textbf{Exploración espacial:} en el proyecto de exploración Mars Rover
  (Mars Exploration Rover Mission) \cite{AplicacionesRovers} tiene como objetivo
  explorar la superficie de Marte en busca de rocas u otros elementos que
  prueben la existencia de agua.
  \item \textbf{Seguimiento (Tracking):} se hace uso en innumerables
  situaciones de carácter estadístico como contar el número o de en áreas de
  vigilancia y seguridad monitorizando trayectorias.
\end{itemize}

%++++++++++++++++++++++++++++++++++++++++++++++++++++++++++++++++++++++++++++++
